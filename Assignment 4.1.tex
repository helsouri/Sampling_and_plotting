\documentclass[letterpaper, 24pt, final, onecolumn, titlepage] {article}

\usepackage{enumerate}
\usepackage{graphicx}
\usepackage{listings}
\usepackage{color}
\usepackage{setspace}
\usepackage {amsmath}
\usepackage{amssymb}
\usepackage{verbatim}
\usepackage{afterpage}
\usepackage{geometry}
\geometry{
 a4paper,
 total={170mm,257mm},
 left=1cm,
 right=1cm,
 }
\title{ECE 270: Computer Methods in ECE \\
	\vspace{1.5cm}
   		\begin{center}\includegraphics{umlogo} \end{center}
	\vspace{1.5cm}
	\textbf{Assignment \#4} \\
	Sampling and Plotting}
	
\author{Hussein El-Souri}

\date{\today}

\definecolor{dkgreen}{rgb}{0,0.6,0}
\definecolor{gray}{rgb}{0.5,0.5,0.5}
\definecolor{mauve}{rgb}{0.58,0,0.82}

\lstset{frame=tb,
  language=C,
  aboveskip=3mm,
  belowskip=3mm,
  showstringspaces=false,
  columns=flexible,
  basicstyle={\small\ttfamily},
  numbers=none,
  numberstyle=\tiny\color{gray},
  keywordstyle=\color{blue},
  commentstyle=\color{dkgreen},
  stringstyle=\color{mauve},
  breaklines=true,
  breakatwhitespace=true,
  tabsize=3
}

\begin{document}

\maketitle

\doublespacing

\section{Statement of the Problem}

The purpose of this assignment is to create a function called the damped sinu-soid, which is given by:\\
\begin{equation}y = e−axcos(wx)\end{equation}
Also, we will make a plot of a real world function based on underlying function rather than a mathematical formula.

\section{Description of Solution}

To make a plot of the damped sinusoid function, we need to first create a set of x-values and calculate how many x-samples there are. The first function takes in the following arguments,xmin, xmax, steps, and an array.\\
xmin:the minimum x-value
xmax:the maximum x-value
Step: the amount of space between samples
The sampling process will contain the following sequence of values:
\begin{equation}x : x_{min}, x_{min} + step,  x_{min} + 2*{step}, ... , x_{max}\end{equation}
The total number of n-samples will be:
\begin{equation}n=\frac{x_{max}-x_{min}}{2}\end{equation}
The $i^{th}$ sample is given by:
\begin{equation}x : x_{min} + i*step, \ for\ i = 0, 1, 2, ....n\end{equation}
\\\\\\
Next, you will use the damped sinusoid function (1) to get y-values from x-values. The function will take in the following arguments: a value for alpha, a value for w, an x-array, a y-array, and a n-sample.\\
It will then input these x-values into the damped sinusoid function to get y-values.\\
in: the value you want to map\\
$in_{Min}$: the minimum value of your in\\
$in_{Max}$:the maximum value of your in\\
$out_{Min}$:the minimum value of your out\\
$out_{Max}$: the maximum value of your out\\
The first thing this function does is find the slope:
\begin{equation}m=\frac{outMax-outMin}{inMax-inMin}\end{equation}
Next, it finds the intercept:
\begin{equation} b= outMax - m*inMax\end{equation}
Lastly, it plugs the values into slope intercept form:
\begin{equation} out = m *in + b\end{equation}

\section{Testing and Output}

The sample I used for the x-values is from 0 to 10 with a step size of 0.05. The parameters of the y-value are -1 to 1. Alpha is 0.5. The value for w is 10.0.

\begin{center}\includegraphics{Damped} \end{center}

\begin{center}\includegraphics{Amazon} \end{center}
\pagebreak

\section{Code}
\singlespacing

\begin{lstlisting}
	//ofApp.h 4 part a)
   //My Variables

    float step = 0.001;



    float xMin = 0.00;

    float xMax = 10.0;



    float yMin = -1.00;

    float yMax = 1.00;



    float xPixMin = 200;

    float xPixMax = 1000;



    float yPixMin = 600;

    float yPixMax = 200;



    int n; //number of samples



    float x[N_MAX];

    float y[N_MAX];





    float xPix[N_MAX];

    float yPix[N_MAX];







    float xDrawMax[N_MAX];

    float xDrawMin[N_MAX];

    float yDrawMax[N_MAX];

    float yDrawMin[N_MAX];



    //My Functions

    int getXSamples (float xMin, float xMax, float step, float x[]);

    void getDampedCosSample (int n, float x[], float y[], float alpha, float w);

    float map (float in, float inMin, float inMax, float outMin, float outMax);

    void map_vec (int n, float in[], float out[], float inMin, float inMax, float outMin, float outMax);

    void printArray (int dim, float x[], char label[]);
};
//-------------------------------------------------------------------------------

//--ofApp.cpp 4 part a)

#include "ofApp.h"

int ofApp::getXSamples (float xMin, float xMax, float step, float x[])

{

    int i;

    int numSamples;

    numSamples = (xMax - xMin) / step + 1;



    for (i = 0; i < numSamples; ++i)

    {

        x[i] = xMin + i * step;

    }



    return numSamples;

}

void ofApp::getDampedCosSample (in



    for(i = 0; i < n; ++i)

    {

        y[i] = exp(-alpha * x[i]) * cos(w * x[i]);

    }

}

float ofApp::map (float in, float inMin, float inMax, float outMin, float outMax)

{

    float m;

    float b;

    float out;



    //slope

    m = (outMax - outMin) / (inMax - inMin);

    //y-inter

    b = outMax - m * inMax;

    //y = mx + b

    out = m * in + b;



    return out;



}

void ofApp::map_vec (int n, float in[], float out[], float inMin, float inMax, float outMin, float outMax)

{

    int i;



    for (i = 0; i < n; ++i)

    {

        out[i] = map (in[i], inMin, inMax, outMin, outMax);

    }

}

void ofApp::printArray (int dim, float x[], char label[])

{

    int i;



    printf("\n%s: ", label);



    for (i = 0; i < dim; ++i)

    {

        printf("%.2f ", x[i]);

    }

}



//--------------------------------------------------------------

void ofApp::setup()

{

    float xSamples[N_MAX];

    float ySamples[N_MAX];

    float alpha = 0.50;

    float omega = 10.00;



    n = getXSamples(xMin, xMax, step, xSamples);

    printArray(n, xSamples, "X values");



    getDampedCosSample (n, xSamples, ySamples, alpha, omega);

    printArray(n, ySamples, "Y values");



    map_vec(n, xSamples, xPix, xMin, xMax, xPixMin, xPixMax);

    printArray(n, xPix, "XPix");



    map_vec(n, ySamples, yPix, yMin, yMax, yPixMin, yPixMax);

    printArray(n, yPix, "YPix");

}



//--------------------------------------------------------------

void ofApp::update()

{



}



//--------------------------------------------------------------

void ofApp::draw()

{

    ofSetBackgroundColor(255, 255, 255);



    float xIntercept;

    float yIntercept;

    int i;



    for(i = 0; i < n; ++i)

    {

        ofSetBackgroundAuto(false);

        ofSetColor(0, 0, 255);

        ofSetLineWidth(10.0f);

        ofCircle(xPix[i], yPix[i], 1);

    }



    xDrawMax[0] = 1000;

    xDrawMin[0] = 200;

    yDrawMax[0] = 200;

    yDrawMin[0] = 200;



    //code the plot y-axis

    yIntercept = (yMax - yMin) / YSTEP + 1;



    for(i = 0; i < yIntercept; i++)

    {

        ofSetBackgroundAuto(false);



        ofSetColor(169, 169, 169);

        ofSetLineWidth(0.001f);

        ofLine(xDrawMin[0], yDrawMin[i], xDrawMax[0], yDrawMax[i]);



        ofSetColor(0, 0, 0);



        yDrawMax[i+1] = yDrawMax[i] + 50;

        yDrawMin[i+1] = yDrawMin[i] + 50;

    }



    xDrawMax[0] = 200;

    xDrawMin[0] = 200;

    yDrawMax[0] = 200;

    yDrawMin[0] = 600;



    //code the plot x-axis

    xIntercept = (xMax - xMin) / XSTEP + 1;



    for(i = 0; i < xIntercept; i++)

    {

        ofSetBackgroundAuto(false);



        ofSetColor(169, 169, 169);

        ofSetLineWidth(0.001f);

        ofLine(xDrawMin[i], yDrawMin[0], xDrawMax[i], 200);



        ofSetColor(0, 0, 0);



        xDrawMax[i+1] = xDrawMax[i] + 40;

        xDrawMin[i+1] = xDrawMin[i] + 40;

    }



    ofSetColor(0, 0, 0);

    ofDrawBitmapString("The Damped Sinusiod Function: exp(-at)cos(wt)", 425, 150);

    ofSetColor(255, 0, 0);

    ofDrawBitmapString("a: 0.50", 425, 220);

    ofDrawBitmapString("w: 10.00", 725, 220);



    //plot y values

    ofSetColor(0, 0, 0);

    ofDrawBitmapString("y(t)", 200 - 80, 375);



    ofDrawBitmapString(" 1.00", 200 - 45, 200);

    ofDrawBitmapString(" 0.75", 200 - 45, 250);

    ofDrawBitmapString(" 0.50", 200 - 45, 300);

    ofDrawBitmapString(" 0.25", 200 - 45, 350);

    ofDrawBitmapString(" 0.00", 200 - 45, 400);

    ofDrawBitmapString("-0.25", 200 - 45, 450);

    ofDrawBitmapString("-0.50", 200 - 45, 500);

    ofDrawBitmapString("-0.75", 200 - 45, 550);

    ofDrawBitmapString("-1.00", 200 - 45, 600);

    //plot x values

    ofDrawBitmapString("time, t", 600 - 10, 600 + 50);



    ofDrawBitmapString("0.0", 200 - 10, 600 + 25);

    ofDrawBitmapString("0.5", 240 - 10, 600 + 25);

    ofDrawBitmapString("1.0", 280 - 10, 600 + 25);

    ofDrawBitmapString("1.5", 320 - 10, 600 + 25);

    ofDrawBitmapString("2.0", 360 - 10, 600 + 25);

    ofDrawBitmapString("2.5", 400 - 10, 600 + 25);

    ofDrawBitmapString("3.0", 440 - 10, 600 + 25);

    ofDrawBitmapString("3.5", 480 - 10, 600 + 25);

    ofDrawBitmapString("4.0", 520 - 10, 600 + 25);

    ofDrawBitmapString("4.5", 560 - 10, 600 + 25);

    ofDrawBitmapString("5.0", 600 - 10, 600 + 25);

    ofDrawBitmapString("5.5", 640 - 10, 600 + 25);

    ofDrawBitmapString("6.0", 680 - 10, 600 + 25);

    ofDrawBitmapString("6.5", 720 - 10, 600 + 25);

    ofDrawBitmapString("7.0", 760 - 10, 600 + 25);

    ofDrawBitmapString("7.5", 800 - 10, 600 + 25);

    ofDrawBitmapString("8.0", 840 - 10, 600 + 25);

    ofDrawBitmapString("8.5", 880 - 10, 600 + 25);

    ofDrawBitmapString("9.0", 920 - 10, 600 + 25);

    ofDrawBitmapString("9.5", 960 - 10, 600 + 25);

    ofDrawBitmapString("10.0", 1000 - 10, 600 + 25);

}
//-----------------------------------------------------------------------------------//

//-----------------------------------------------------------------------------------//
//-----------------------------------------------------------------------------------//

//-----------------------------------------------------------------------------------//

//ofapp.h  4 part b)

float step = 0.1;



    float xMin = 0.00;

    float xMax = 168.1;



    float yMin = 0;

    float yMax = 350;



    float xPixMin = 1000;

    float xPixMax = 400;



    float yPixMin = 600;

    float yPixMax = 200;



    int n; //number of samples



    float xPix[N_MAX];

    float yPix[N_MAX];



    //My Functions

    int getXSamples (float xMin, float xMax, float step, float x[]);

    void getDampedCosSample (int n, float x[], float y[], float alpha, float w);

    float map (float in, float inMin, float inMax, float outMin, float outMax);

    void map_vec (int n, float in[], float out[], float inMin, float inMax, float outMin, float outMax);

    void printArray (int dim, float x[], char label[]);

//---------------------------------------------------------------------------------------------------

//---------------------------------------------------------------------------------------------------
//ofApp.cpp 4 paart b)

#include "ofApp.h"

int ofApp::getXSamples (float xMin, float xMax, float step, float x[])

{

    int i;

    int numSamples;

    numSamples = (xMax - xMin) / step + 1;



    for (i = 0; i < numSamples; ++i)

    {

        x[i] = xMin + i * step;

    }



    return numSamples;

}

void ofApp::getDampedCosSample (int n, float x[], float y[], float alpha, float w)

{

    int i;



    for(i = 0; i < n; ++i)

    {

        y[i] = exp(-alpha * x[i]) * cos(w * x[i]);

    }

}

float ofApp::map (float in, float inMin, float inMax, float outMin, float outMax)

{

    float m;

    float b;

    float out;



    //slope

    m = (outMax - outMin) / (inMax - inMin);

    //y-inter

    b = outMax - m * inMax;

    //y = mx + b

    out = m * in + b;



    return out;



}

void ofApp::map_vec (int n, float in[], float out[], float inMin, float inMax, float outMin, float outMax)

{

    int i;



    for (i = 0; i < n; ++i)

    {

        out[i] = map (in[i], inMin, inMax, outMin, outMax);

    }

}

void ofApp::printArray (int dim, float x[], char label[])

{

    int i;



    printf("\n%s: ", label);



    for (i = 0; i < dim; ++i)

    {#include "ofApp.h"

int ofApp::getXSamples (float xMin, float xMax, float step, float x[])

{

    int i;

    int numSamples;

    numSamples = (xMax - xMin) / step + 1;



    for (i = 0; i < numSamples; ++i)

    {

        x[i] = xMin + i * step;

    }



    return numSamples;

}

void ofApp::getDampedCosSample (int n, float x[], float y[], float alpha, float w)

{

    int i;



    for(i = 0; i < n; ++i)

    {

        y[i] = exp(-alpha * x[i]) * cos(w * x[i]);

    }

}

float ofApp::map (float in, float inMin, float inMax, float outMin, float outMax)

{

    float m;

    float b;

    float out;



    //slope

    m = (outMax - outMin) / (inMax - inMin);

    //y-inter

    b = outMax - m * inMax;

    //y = mx + b

    out = m * in + b;



    return out;



}

void ofApp::map_vec (int n, float in[], float out[], float inMin, float inMax, float outMin, float outMax)

{

    int i;



    for (i = 0; i < n; ++i)

    {

        out[i] = map (in[i], inMin, inMax, outMin, outMax);

    }

}

void ofApp::printArray (int dim, float x[], char label[])

{

    int i;



    printf("\n%s: ", label);




        printf("%.2f ", x[i]);

    }

}



//--------------------------------------------------------------

void ofApp::setup()

{

    float xSamples[N_MAX];

    float xData[N_MAX];

    int i;

    int k;

    int j;

    int isReading;

    FILE *fp;



    fp = fopen("C:\\Users\\shawn\\Google Drive\\Programming\\OpenFramworks\\examples\\Assignments\\7 assignment 4.2\\bin\data\\TeslaData.csv", "r");



    isReading = 1;

    j = 0;



    while(isReading == 1)

    {

        k = fscanf(fp, "%f", &xData[j]);



        if (k == EOF)

            isReading = 0;



        else

            j++;

    }



    //printArray(j, xData, "Data values");



    n = getXSamples(xMin, xMax, step, xSamples);

   // printArray(n, xSamples, "X values");



    map_vec(n, xSamples, xPix, xMin, xMax, xPixMin, xPixMax);

    //printArray(n, xPix, "XPix");



    map_vec(n, xData, yPix, yMin, yMax, yPixMin, yPixMax);

    //printArray(n, yPix, "YPix");



    fclose(fp);

}



//--------------------------------------------------------------

void ofApp::update()

{



}



//--------------------------------------------------------------

void ofApp::draw()

{

    ofSetBackgroundColor(255, 255, 255);



    int i;



    for(i = 0; i < n; ++i)

    {

        ofSetBackgroundAuto(false);

        ofSetColor(0, 0, 255);

        ofSetLineWidth(10.0f);

        ofFill();

        ofCircle(xPix[i], yPix[i], 1);

    }



    ofSetColor(169, 169, 169);

    ofSetLineWidth(0.001f);


    ofSetColor(0, 0, 0);

    ofDrawBitmapString("Tesla (TSLA)", 600, 275);

    ofDrawBitmapString("Price", 275, 450);

    ofDrawBitmapString("Year", 750, 650);



    ofDrawBitmapString("300", 350, 255);

    ofDrawBitmapString("275", 350, 285);

    ofDrawBitmapString("250", 350, 315);

    ofDrawBitmapString("225", 350, 345);

    ofDrawBitmapString("200", 350, 375);

    ofDrawBitmapString("175", 350, 405);

    ofDrawBitmapString("150", 350, 435);

    ofDrawBitmapString("125", 350, 465);

    ofDrawBitmapString("100", 350, 495);

    ofDrawBitmapString("75", 350, 525);

    ofDrawBitmapString("50", 350, 555);

    ofDrawBitmapString("25", 350, 585);

    ofDrawBitmapString("0", 350, 615);



    ofDrawBitmapString("2011", 450, 625);

    ofDrawBitmapString("2012", 530, 625);

    ofDrawBitmapString("2013", 610, 625);

    ofDrawBitmapString("2014", 690, 625);

    ofDrawBitmapString("2015", 770, 625);

    ofDrawBitmapString("2016", 850, 625);

    ofDrawBitmapString("2017", 930, 625);

}


\end{lstlisting}

\end{document}